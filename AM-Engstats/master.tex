Chapter 2 Probability 2.1 Introduction to Probability 
Probability theory is the mathematical study of randomness. A probability model of a random experiment is defined by assigning probabilities to all the different outcomes. The set of all possible outcomes from the experiment is known but the actual outcome is not known in advance. The set of all possible outcomes is called the sample space and it is usually denoted by the letter S. 
An event, A is a subset of the sample space of a random experiment. When the experiment is performed, the event A is said to occur if the outcome of the experiment lies in the set A. Examples: (a) Toss a coin. The set of possible outcomes is S=Head, Tail and the event of obtaining a head when tossing a coin is represented by the subset A—Head. (b) Roll a die. The set of possible outcomes is S=1,2,3,4,5,6 and the event of getting an even number is represented by an event A-2,4,6. 
26 



%==============================%

Example A survey of magazine subscribers showed that 45.8 percent rented a car during the past 12 months for business reasons, 54 percent rented a car for personal reasons, and 30 percent rented a car during the past 12 months for both business and personal reasons. (a) What is the probability that a subscriber rented a car during the past 12 months for business or personal reasons? (b) What is the probability that a subscriber did not rent a car during the past 12 months for either business or personal reasons? 
2.3 Conditional Probability 
Consider a class consisting of 100 students 50 of whom are male and 50 female. Suppose the 65 of the students are from Munster and 35 are not and that 27 of the males are from Munster. Consider the random experiment which involves putting the names of all 100 students into a hat and drawing one out at random. (a) What is the probability that the person chosen is a female? (b) What is the probability that the person chosen is from Munster? (c) What is the probability that the person chosen is a female student from Munster? (d) Suppose that the person chosen is from Munster. What is the probability that the person chosen is a female? 
Definition: The conditional probability that event A occurs given that event B has already 
occurred is given by P(AJB). 
30 






%=============================%

Example A survey of magazine subscribers showed that 45.8 percent rented a car during the past 12 months for business reasons, 54 percent rented a car for personal reasons, and 30 percent rented a car during the past 12 months for both business and personal reasons. (a) What is the probability that a subscriber rented a car during the past 12 months for business or personal reasons? (b) What is the probability that a subscriber did not rent a car during the past 12 months for either business or personal reasons? 
2.3 Conditional Probability 
Consider a class consisting of 100 students 50 of whom are male and 50 female. Suppose the 65 of the students are from Munster and 35 are not and that 27 of the males are from Munster. Consider the random experiment which involves putting the names of all 100 students into a hat and drawing one out at random. (a) What is the probability that the person chosen is a female? (b) What is the probability that the person chosen is from Munster? (c) What is the probability that the person chosen is a female student from Munster? (d) Suppose that the person chosen is from Munster. What is the probability that the person chosen is a female? 
Definition: The conditional probability that event A occurs given that event B has already 
occurred is given by P(AJB). 
30 


%=====================%



Example The probability that a car battery subject to high engine compartment temperature suffers low charging current is 0.7. The probability that a battery is subject to high engine compartment temperature is 0.05. What is the probability that a battery is subject to low charging current and high engine compartment temperature? 
Using the Multiplication Law , we can re-write the conditional probabilities as: 
P(AIB) = P(B)14)Bc(A) P(BIA) = P(AipipAI;(13) 
The relationship between the two conditioning probabilities is called Bayes'Theorem. 
Total Probability Law 
P(13) = P(B n + FIB n74) 
Example Suppose that in semiconductor manufacturing the probability is 0.1 that a chip that is subject to high levels of contamination during manufacturing causes a product failure. The probability is 0.005 that a chip that is not subjected to high contamination levels during manufacturing causes a product failure. In a particular production run, 20% of the chips are subject to high levels of contamination. What is the probability that a product using one of these chips fail? Let F denote the event that the product fails, and let A denote the event that the chip is exposed to high levels of contamination. The requested probability is P(F), and the information provided can be represented as: P(FIA) = 0.1, P(PIA) = 0.005 and P(A)-0.2 
32 


%=============================%



Example: The probabilities for events Al and A2 are P(A1)=0.40 and P(A2)-0.60. It is also known that P(A1 il A2)=0. Suppose P(BIA1) -= 0.20 and P(BIA2) = 0.05. 
1. Are Al and A2 mutually exclusive? 2. Compute P(A1 fl B) and P(A2 n B). 3. Calculate 13(3). 4. Apply Bayes'Theorem to compute P(A1 IB) and P(A2IB). 
Example: 10% of people in a population have a certain disease. Consider a diagnostic test, the outcome of which is either a positive or negative result. Suppose that 85% of people with the disease produce a positive result while only 5% of people without the disease produce a positive result. 
1. What percentage of the population will produce a positive result? 
2. What is the probability that a person who produces a positive result does indeed have the 
disease? 
3. What is the probability that a person who produces a negative result does not have the disease? 
3-1 

%================================%

2.4 Questions 
Qi. In a study of 100 students who had been awarded university scholarships, it was found that 40 had part-time jobs, 25 had made the dean's list the previous semester, and 15 had both a part-time job and had made the dean's list. What was the probability that a student had a part-time job or was on the dean's list? 
Q2. A survey of subscribers to Fortune magazine showed that 54% rented a car in the past 12 months for business reasons, 51% rented a car for personal reasons, and 72% rented a car for either business or personal reasons. (a) What is the probability a subscriber rented a car in the past 12 months for business reasons and for personal reasons? (b) What is the probability a subscriber did not rent a car in the past 12 months? 
Q3. There are 8 white balls and two black balls in urn 1, and there are 9 black balls and 1 white ball in urn 2. One ball is picked from one of the urns. The ball will be picked from urn 1 if the outcome of a rolling fair dice is 1 or 2, otherwise, the ball will be picked from urn 2. Given that the colour of the ball that was picked is white, what is the probability that it was picked from urn 
2. 
Q4. Electric motors coming off two assembly lines are pooled for storage in a common stock-room, and the room contains an equal number of motors from each line. Motors are periodically sampled from that room and tested. It is known that 6% of the motors from line 1 are defective and 9% of the motors from line 2 are defective. If a motor is randomly selected from the stockroom and found to be defective, find the probability that it came from line 2. 
36 
%===============================%
3.3 Continuous Probability Distributions 
3.3.1 Introduction 
A continuous random variable may assume any value in an interval on the real line or in a collection of intervals. Since any interval contains an infinite number of values, it is not possible to talk about the probability that the random variable will assume a specific value; instead, we must think in terms of the probability that a continuous random variable will assume a value within a given interval. 
For example, consider the random experiment of picking a person at random from the class and let X denote the random variable corresponding to the blood pressure of the person chosen. Keep in mind that the blood pressure is not a constant property of an individual but changes over time and may change substantially even over a short period of time. Clearly the smallest value that X can take is 0 = which would be bad news for the person chosen! But what is the next smallest value that X could take? We cannot say and therefore we cannot write down a list of possible values for X. We might still, however, be interested in calculating a probability such as, for example, the probability that X is greater than 100mm/Hg (mm/Hg stands for millimeters of mercury, which are the units in which blood pressure is measured) which we write as P[X > 100]. 
3.3.2 The Notion of a Density Function 
Let X be a continuous random variable. Suppose we repeat the random experiment which produces X a total of N times and draw a histogram using intervals of length d. Now imagine what would happen if we allowed N - the number of repetitions - to tend to oo and d - the length of the interval, to tend to zero. What we would get is a curve having the property that the total area under the curve is equal to 1 and the area under the curve between two points a and b is the proportion of 
54 

%===============================%
\cahpter{distributions}

Example: Let the continuous random variable X denote the diameter of a hole drilled in a sheet metal component. The target diameter is 12.5 millimetres. Most random disturbances to the process result in larger diameters. Historical data show that the distribution of X can be modelled by a probability density function f(2:) = 20e-20(' 12.5) x > 12.5. 
(a) If a part with diameter larger than 12.6 millimetres is scrapped, what proportion of parts is scrapped? (b) What proportion of parts is between 12.5 and 12.6 millimetres? 
Mean and Variance of a Continuous Random Variable Suppose X is a continuous random variable with probability density function f(x) The expected value of X is: 
E(X) = u = f x f (x)dx 
The variance of X is: 
V ar(X) = Cr2 = f (x p)2f(x)dx - f X2 f (x)dx - p2 

% End of page 56
%=======================================%

Example: 
For the previous example, calculate the mean and the variance of X. 
3.3.3 The Uniform Distribution 
Motivating Example: Suppose that the flight time of an aeroplane travelling from Chicago to New York can be any value in the interval from 120 to 140 minutes. Assume that the probability of a flight time within any 1-minute interval is the same as the probability of a flight time within any other 1-minute interval up to and including 140 minutes. Let X denote the total flight time. Then, with every 1-minute interval being equally likely, the random variable X is said to have a uniform probability distribution. 
Example: 
What is the probability of a flight time between 130 and 140 minutes? 
In general, let X be a continuous random variable whose density function is given by: 
f (s) =1 -6---a for a < x < b 
= 0 otherwise 
where a < b are given constants. Then X is said to have uniform density on the interval [a, bi. 
57 
%=======================================%
Example: 
Let X be a random variable which has a uniform density on the interval [0, 11. Calculate: 
(a) P10.2 < X < 0.6) (b) E[X] (c) Var[X) 
Expected Value and Variance for the Uniform Random Distribution 
E(X) = gik Var(X) = .72— b 12 
3.3.4 The Normal Probability Distribution 
Undoubtedly, the most widely used model for the distribution of a random variable is a normal distribution. The form, or shape of the normal probability distribution is the bell-shaped curve. The probability density function that defines the bell-shaped curve of the normal probability distribution is given by: 
1 f (x) = —e C. 2-s)2 for oo < x < oo Q VT-7r 
where p = expected value, or mean of the random variable X. a -- standard deviation of the random variable X. 
58 

%====================================%

The mathematical shorthand for saying that a variable X is normally distributed with a mean 
of it and a variance of 172 is X ,..--, NOI,a2). 

a 
ii 
A 
Y 
Figure 3.2: Probability density function. 
Characteristics of the Normal Probability Distribution: 1. The highest point on the normal curve is at the mean, which is also the median and mode of the distribution. 2. The mean of the distribution can be any numerical value: negative, zero or positive. 3. The normal probability distribution is symmetric, with the shape of the curve to the left of the mean being a mirror image of the shape of the curve to the right of the mean. 4. The standard deviation determines the width of the curve. Larger values of the standard devia-tion result in wider, flatter curves, showing more dispersion in the data. 5. The total area under the curve for the normal probability distribution is 1. 6. Probabilities for the normal random variables are given by areas under the curve. Probabilities for some commonly used intervals are: 
59 


%====================================%
• 68% of the time, a normal random variable assumes a value within plus or minus one standard 
deviation of its mean. 
• 95% of the time, a normal random variable assumes a value within plus or minus 1.96 times 
the standard deviation of its mean. 
• 99% of the time, a normal random variable assumes a value within plus or minus 2.58 times 
the standard deviation of its mean. 
An example of a normal probability distribution is IQ score of adults. The mean (p) is equal to about 100 and the standard deviation (a) is equal to about 15. The IQ of most people peaks close to the mean of 100 and the proportions of very high or very low scores are relatively small. The distribution is symmetrical about the mean: 
• The proportion of people scoring below the mean is equal to the proportion scoring above the 
mean: that is, the median is equal to the mean. 
• The proportion of people within the interval A ± a, that is 85 to 115, is approximately 0.68 or 
68%. Therefore 16% of people score below 85 and 16% of people score above 115. 
• The proportion of people within the interval p ± 1.96a, that is 70.6 to 129.4, is 0.95 or 95%. 
Therefore 2.5% of people score lower than 70.6 and 2.5% of people score higher than 129.4. 
• The proportion of people within the interval p ± 2.58a, that is 61.3 to 138.7, is 0.99 or 99%. 
Therefore 0.5% of people score lower than 61.3 and 0.5% of people score higher than 138.7 
We will use statistical tables to find the probability that a variable will have a value in a given interval. We can have normal distributions with many different means and variances. Instead of having a statistical table for each one, we need to convert the distribution we have into a standard 
60 

%===============================%
one for which there is a statistical table available. This distribution is called the standard normal 
probability distribution. 
The Standard Normal Probability Distribution A random variable that has a normal distribution with a mean of zero and a standard deviation of one is said to have a standard normal probability distribution. When we have a normal distribution with any mean p and any standard deviation a, we answer probability questions about the distribution by first converting to the standard normal distribution. The formula used to convert any normal random variable X with mean p and standard deviation a to the standard normal distribution is given by: 
— p —  a 
Note: 1. Z is a standard normal random variable with a mean of 0 and a standard deviation of 1. 2. Z can be thought of as a measure of the number of standard deviations x is from it. 
For example, 
• A value of x equal to its mean p results in z = — 0; thus, we see that a value of x equal 
to its mean p corresponds to a value of z at its mean 0. 
• A value of x that is one standard deviation above its mean; x = p + a, we see that the 
corresponding z value is z = 1(+c)-Y1 = = 1. Thus a value that is one standard deviation a 
above its mean yields z = 1. 
Areas in the tail of the normal distribution table: 
Table 3 in Murdoch and Barnes gives the probabilities for a standard normal density. The table 
61 
%===============================%
